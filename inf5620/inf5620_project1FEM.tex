\documentclass[a4paper,10pt]{article}
\usepackage[utf8]{inputenc}
\usepackage{parskip}
\usepackage{amsmath, amssymb}

\title{1D Wave equation with finite elements}
\author{Yapi Donatien Achou}


\begin{document}
\maketitle

\begin{description}
\item[a)]

We will construct a difference equation for all coefficients $c^{n+1}_j$, so we need to determine the initial values $c^0_j$. 
The first thing we do is to find an expression for the residual at time step 0, ie. the measure of the error in fulfilling the governing equation. 
The residual at time 0 is given by
\begin{multline*}
 R^0 = \mathcal{L} (u^0) = \mathcal{L} (\sum_j c_j^0 \psi_j^0 ), \text{ where } \\
 \mathcal{L} (u) = u_{tt} -c^2 u_{xx} \text{ according to our problem model.}
 \end{multline*}
We will look at two principles for making R small: the Galerkin method and the collocation method. But first we find an expression for R:

The initial conditions for $t = 0$ are given by 
\begin{equation*}
 u(x,0) = I(x) \text{ and } u_t(x,0) = V(x).
 \end{equation*}

Hence we can express 
\begin{equation*}
 u_{xx}^0 = I_{xx} (x) \text{ and } u_{tt}^0 = V_x(x)= 0.
\end{equation*}

Inserting these results into the expression for $R^0$ we have
\[ R^0 =  -c^2 u_{xx}^0 = -c^2 I_{xx}  \]

Hence we start using the Galerkin method: we seek ${c_i}$ such that $\langle R,v \rangle = 0$. We choose $v(x) = \phi_i (x)$. 
Multiply by a function $v \in V$ and integrate on both sides, using integration by parts (the coefficient $c^2$ falls):
\begin{equation*}
 \int_0^L I_{xx} v \mathrm{d}x = 0 \qquad \to \qquad
 - \int_0^L I_x v_x \mathrm{d}x = 0 
\end{equation*}
We seek a solution $u$ of the form
\[ u(x, t^n) = U_0 (t_n) \phi_0 (x) + \sum_{j=0}^{N_n} c_j^n \phi_{j+1} (x) \]
If we then let  $t_n = 0$, we have 
\[ \underbrace{u(x, 0)}_{I(x)} = \underbrace{ U_0 (0) }_{I(0)} \phi_0 (x) + \sum_{j=0}^{N_n} c_j^0 \phi_{j+1} (x) \]
Hence we can derivate the expression for $I_x(x)$ and insert it into our previous result, together with the chosen $v(x)$:
\begin{align*}
 I_x(x) = U_0 (0) \phi_0^{'} (x) + \sum_j c_j^0 \phi_{j+1}^{'} (x) \\ v_x(x) = \phi_i^{'} (x)\\ 
 \Rightarrow \int_0^L \Big( U_0 (0) \phi_0^{'} (x) + \sum_j c_j^0 \phi_{j+1}^{'} (x) \Big) \phi_i^{'} (x) \mathrm{d}x = 0 \\
 \Rightarrow \underbrace{ \int_0^L \phi_{j+1}^{'} (x) \phi_i^{'} (x) \mathrm{d}x }_{ A_{i,j}^0 } \underbrace{ \sum_{j=0}^{N_n} c_j^0 }_{c_j^0} 
 = \underbrace{ - \int_0^L U_0 (0) \phi_0^{'} (x) \phi_i^{'} (x) \mathrm{d}x}_{ b_{i,j}^0} \\
 \end{align*}
We have thus derived a linear system to find $c_j^0$.

If we instead use the collocation method, we must start choosing a set of $N+1$ collocation points uniformly distributed over 
the interior of our domain: let $x_i = i \Delta x$, $\Delta x = L/N$, $i = 0,1,...,N-1$. We thus demand the residual to vanish at these points, 
ie. we demand the error to be zero at a finite set of values. \newline
The solution form 
\[ u(x, 0) = I(x) = U_0 (0) \phi_0 (x) + \sum_{j=0}^{N_n} c_j^0 \phi_{j+1} (x)\]
is directly plugged in the problem:
\[ -\sum_j c_j^0 \phi_{j+1}^{''} (x) = I(0) \phi_0 (x) \]
and the collocation procedure leads to the equations
\begin{align*}
 \underbrace{\phi_{j+1}^{''}(x_i)}_{A_{ij}} \underbrace{\sum_j c_j^0}_{c_j} = \underbrace{I(0) \phi_0 (x_i)}_{b_i} \text{,}
\end{align*}
Again we have obtained a set of equations to find $c_j^0$. 

\vspace{1cm}

\item[b)]
The wave equation:
\[ u_{tt} = c^2 u_{xx} \]
We start deriving a finite difference approximation for the left side with respect to time:
\[ u_{tt} \approx \frac{u^{n-1} - 2 u ^n + u^{n+1} }{ \Delta t^2 } \]
Inserting this approximation into the equation, we can solve for the next time step: we have a formulation for the sequence of spacial problems
given by
\[ u^{n+1} (x) = \Delta t^2 c^2 u_{xx} ^n (x) - u^{n-1} (x) + 2 u ^n (x) \]
The first time step must be considered as a special case since we do not have a value for $u^{-1}$. We use the discretized version of 
the initial condition over the derivative of $u$ implying $\frac{\delta u}{\delta t}(x,0) = V(x)$:
\[ \frac{u^{n-1} + u^{n+1}}{2 \Delta t} = V (x) \]
and this gives
\[ u^{n-1} = 2 \Delta t V (x) - u^{n+1}. \]
Insert:
\[ u^{n+1} = \Delta t^2 c^2 u_{xx}^n (x) - (2 \Delta t V - u^{n+1} (x) ) + 2 u^n (x) \]
Hence the scheme for the first time step becomes 
\[ u^1 = \frac{1}{2} \Delta t^2 c^2 u_{xx}^0 (x) + u^0 (x) - \Delta t V (x) \]
 
\vspace{1cm}
\item[c)]
The first step is finite difference differentiation respect to time: 
\[ \frac{ u^{n+1} -2u^n + u^{n-1} }{\Delta t^2} = c^2 u_{xx}^n \]
Solving for $u^{n+1}$ we get
\[ u^{n+1} = \Delta t^2 c^2 u_{xx}^n + 2 u^n - u^{n-1} \]
The Galerkin method gives rise to the equation
\begin{align*}
 \int_0^L u^{n+1} v \mathrm{d} x = \Delta t^2 c^2 \underbrace{ \int_0^L u_{xx}^n v \mathrm{d} x }_{ = \lbrack v u_x \rbrack_0^L - \int v_x u_x = 
 - \int v_x u_x } +  2 \int_0^L u^n v \mathrm{d} x - \int_0^L u^{n-1} v \mathrm{d} x \\
 = - \Delta t^2 c^2 \int v_x u_x + 2 \int_0^L u^n v \mathrm{d} x - \int_0^L u^{n-1} v \mathrm{d} x \\
\end{align*}
This is equivalent to
\begin{align*}
\underbrace{ \langle u^{n+1},v \rangle}_{ a(u,v) } = \underbrace{ -\langle \Delta t^2 c^2 v_x, u_x^n \rangle + \langle 2 u^n, v \rangle -
\langle u^{n-1}, v \rangle}_{L(v)}
\end{align*}

Thus our problem can be formulated in a variational form for general $u$ and $v$: \newline 
\emph{Find u such that} $a(u,v) = L(v) \forall v \in V$, where
\begin{align*}
a(u,v) = \langle u^{n+1},v \rangle \qquad L(v) = -\langle \Delta t^2 c^2 v_x, u_x^n \rangle + \langle 2 u^n, v \rangle - \langle u^{n-1}, v \rangle
\end{align*}

\vspace{1cm}
\item[d)]
In order to derive the linear system that will find $c_j$ at each time step, we must express the time dependency through a finite 
difference approximation:
\[ u_{tt} \approx \frac{u^{n-1} - 2 u^n + u^{n+1}}{\Delta t^2} \]
We let $t_n = n \Delta t$ and seek a finite element solution of the form \newline $u(x, t^n) = u^n (x) = \sum_{j=0}^{N_n} c_j^n \phi_j (x)$. 
\newline Inserting the approximation into the problem, we  get an expression for $u^{n+1}$:
\[ u^{n+1} = 2 u^n - u^{n-1} + \Delta t^2 c^2 u_{xx} \]
Again we proceed with the Galerkin method: introduce a test function $v \in V = \mathrm{Span} \{ \phi_0, \phi_1, {\dots}, \phi_n \}$
and integrate over the spacial mesh $[0, L]$:
\begin{multline*}
 \int_0^L u^{n+1} v \mathrm{d} x = 2 \int_0^L u^n v \mathrm{d} x - 
 \int_0^L u^{n-1} v \mathrm{d} x + \Delta t^2 c^2 \int_0^L u^n_{xx} v \mathrm{d} x \\
\text{using integration by parts:} \\
= 2 \int_0^L u^n v \mathrm{d} x - \int_0^L u^{n-1} v \mathrm{d} x - \Delta t^2 c^2 \int_0^L u^n_x v_x \mathrm{d} x \\
\text{Finally let $v (x) = \phi_i (x) $ and insert, together with the expression for $u^n (x)$: } \\
\underbrace{ \int_0^L \phi_i \phi_j \mathrm{d} x}_{M_{i,j}} \sum c_j^{n+1} = \\ 2 \underbrace{ \int_0^L \phi_i \phi_j \mathrm{d} x }_{M_{i,j}} 
\sum c_j^n - \underbrace{ \int_0^L \phi_i \phi_j \mathrm{d} x}_{M_{i,j}} \sum c_j^{n-1} - 
\Delta t^2 c^2 \underbrace{ \int_0^L \phi_i^{'} \phi_j^{'} \mathrm{d} x }_{K_{i,j}} \sum c_j^n \\
\end{multline*}

We have thus derived the linear system to be solved for finding all $c_j^n$:
\[ M c^{n+1} = 2M c^n - M c^{n-1} - \Delta t^2 c^2 K c^n\]

\newpage 
\item[e)]
We assume further that we are using P1 elements. How will $M$ and $K$ look like? In general, we have from d) that
\[ M_{i,j} = \int_{x_i}^{x_{i+1}} \phi_i \phi_j \mathrm{d} x \qquad K_{i,j} = \int_{x_i}^{x_{i+1}} \phi_i^{'} \phi_j^{'} \mathrm{d} x \]
We also have that, on the reference element $[-1,1]$,
\[ \phi_0(X) = 1/2 (1-X) \qquad \phi_1(X) = 1/2 (1+X) \]
\[ \phi_0^{'}(X) = -1/2 \qquad \phi_{'} = 1/2 \]
so
\[ M_{i,j}^e = h/2 \int_{-1}^1 \phi_i \phi_j \mathrm{d} X \qquad K_{i,j}^e = h/2 \int_{-1}^{1} \phi_i^{'} \phi_j^{'} \mathrm{d} X \qquad i,j = 0,1 \]
resulting in the matrices
\begin{equation*}
 M^e = \frac{h}{6}
 \begin{bmatrix}
  2 & 1\\
  1 & 2
 \end{bmatrix}
 \qquad
 K^e = \frac{h}{4}
 \begin{bmatrix}
  1 & -1\\
  -1 & 1
 \end{bmatrix}
\end{equation*}
Assembly to global matrices gives:
\begin{equation*}
 M = \frac{h}{6}
 \begin{bmatrix}
  2 & 1 \\
  1 & 4 & 1 \\
  & & \cdots \\
  & & 1 & 4 & 1 \\
  & & & 1 & 2 \\
 \end{bmatrix}
 \qquad
 K = \frac{h}{4}
 \begin{bmatrix}
  1 & -1 \\
  -1 & 2 & -1 \\
  & & \cdots \\
  & & -1 & 2 & -1 \\
  & & & -1 & 1 \\
 \end{bmatrix}
\end{equation*}



\item[f)]
From d) and e) we have that a typical row in the matrix system
\[ M c^{n+1} = 2M c^n - M c^{n-1} - \Delta t^2 c^2 K c^n\]
can be written as 
\begin{multline*}
 \frac{h}{6} \Big( c_{i-1}^{n+1} + 4 c_{i}^{n+1} + c_{i+1}^{n+1} \Big) = \\
 2 \frac{h}{6} \Big( c_{i-1}^{n} + 4 c_{i}^{n} + c_{i+1}^{n} \Big) -
 \frac{h}{6} \Big( c_{i-1}^{n-1} + 4 c_{i}^{n-1} + c_{i+1}^{n-1} \Big) - 
 \Delta t^2 c^2 \frac{h}{4} \Big( - c_{i-1}^{n} + 2 c_{i}^{n} - c_{i+1}^{n} \Big)
 \end{multline*}
Since $u(x_j) = \sum_j c_ \phi (x_j) ) \sum_j c_j \delta_{ij} = c_j$, the unknowns $c_0,...,c_N$ are actually $u_1,...,u_{Nn}$.  
Hence with a shift of indexes we can insert $u_i = c_{i-1}$ and we get
\begin{multline*}
 \frac{h}{6} \Big( u_{i-1}^{n+1} + 4 u_{i}^{n+1} + u_{i+1}^{n+1} \Big) = \\
 2 \frac{h}{6} \Big( u_{i-1}^{n} + 4 u_{i}^{n} + u_{i+1}^{n} \Big) -
 \frac{h}{6} \Big( u_{i-1}^{n-1} + 4 u_{i}^{n-1} + u_{i+1}^{n-1} \Big) - 
 \Delta t^2 c^2 \frac{h}{4} \Big( - u_{i-1}^{n} + 2 u_{i}^{n} - u_{i+1}^{n} \Big)
 \end{multline*}
 
Next we let $h = \Delta x$.\newline 
In general, we have that
\[ \lbrack D_t D_t u \rbrack_i^n = \frac{u_i^{n+1} - 2 u_i^n + u_i^{n-1}}{\Delta t^2} \qquad
 \lbrack D_x D_x u \rbrack_i^n = \frac{u_{i+1}^{n} - 2 u_i^n + u_{i-1}^{n}}{\Delta x^2} \]
so
\[ \lbrack D_t D_t (D_x D_x u) \rbrack_i^n = \frac{\frac{u_{i+1}^{n+1} - 2 u_i^{n+1} + u_{i-1}^{n+1}}{\Delta x^2} 
- 2 \frac{u_{i+1}^{n} - 2 u_i^n + u_{i-1}^{n}}{\Delta x^2} + \frac{u_{i+1}^{n-1} - 2 u_i^{n-1} + u_{i-1}^{n-1}}{\Delta x^2} }{\Delta t^2} \]

Let us now try to reformulate our problem and achieve a compact operator description:
\begin{multline*}
 \frac{\Delta x}{6} \Big( u_{i-1}^{n+1} + 4 u_{i}^{n+1} + u_{i+1}^{n+1} \Big) = \\
 2 \frac{\Delta x}{6} \Big( u_{i-1}^{n} + 4 u_{i}^{n} + u_{i+1}^{n} \Big) -
 \frac{\Delta x}{6} \Big( u_{i-1}^{n-1} + 4 u_{i}^{n-1} + u_{i+1}^{n-1} \Big) - 
 \Delta t^2 c^2 \frac{\Delta x}{4} \Big( - u_{i-1}^{n} + 2 u_{i}^{n} - u_{i+1}^{n} \Big)\\
 \Downarrow \\
 \frac{\Delta x}{6} \Big( \Big( u_{i-1}^{n+1} + 4 u_{i}^{n+1} + u_{i+1}^{n+1} \Big) -
 2 \Big( u_{i-1}^{n} + 4 u_{i}^{n} + u_{i+1}^{n} \Big) +
 \Big( u_{i-1}^{n-1} + 4 u_{i}^{n-1} + u_{i+1}^{n-1} \Big) \Big) = \\ 
 \Delta t^2 c^2 \frac{\Delta x}{4} \Big( - u_{i-1}^{n} + 2 u_{i}^{n} - u_{i+1}^{n} \Big)\\
\end{multline*}
Subtracting $\Delta x(u_i^{n+1} - 2 u_i^n + u_i^{n-1})$ from both sides we get 
\begin{multline*}
 \frac{\Delta x}{6} \Big( ( u_{i-1}^{n+1} - 2 u_{i}^{n+1} + u_{i+1}^{n+1} ) -
 2 ( u_{i-1}^{n} - 2 u_{i}^{n} + u_{i+1}^{n} ) +
( u_{i-1}^{n-1} - 2 u_{i}^{n-1} + u_{i+1}^{n-1} ) \Big) = \\ 
 \Delta t^2 c^2 \frac{\Delta x}{4} \Big( - u_{i-1}^{n} + 2 u_{i}^{n} - u_{i+1}^{n} \Big) - \Delta x \Big( u_i^{n+1} - 2 u_i^n + u_i^{n-1} \Big)\\
\\
 \Updownarrow \\
 \\
 \frac{ \frac{\Delta x}{6} \Big( ( u_{i-1}^{n+1} - 2 u_{i}^{n+1} + u_{i+1}^{n+1} ) -
 2 ( u_{i-1}^{n} - 2 u_{i}^{n} + u_{i+1}^{n} ) +
( u_{i-1}^{n-1} - 2 u_{i}^{n-1} + u_{i+1}^{n-1} ) \Big) }{\Delta t^2} = \\ 
 c^2 \frac{\Delta x}{4} \Big( - u_{i-1}^{n} + 2 u_{i}^{n} - u_{i+1}^{n} \Big) - \frac{\Delta x}{\Delta t^2} \Big( u_i^{n+1} - 2 u_i^n + u_i^{n-1} \Big)\\ 
\\
 \Updownarrow \\
\\
\frac{1}{6} \frac { \Big( ( \frac{u_{i-1}^{n+1} - 2 u_{i}^{n+1} + u_{i+1}^{n+1}}{\Delta x^2} ) -
 2 ( \frac{u_{i-1}^{n} - 2 u_{i}^{n} + u_{i+1}^{n}}{\Delta x^2} ) +
( \frac{ u_{i-1}^{n-1} - 2 u_{i}^{n-1} + u_{i+1}^{n-1}}{\Delta x^2} ) \Big) }{\Delta t^2} = \\ 
 c^2 \frac{1}{4} \Big( \frac{- u_{i-1}^{n} + 2 u_{i}^{n} - u_{i+1}^{n}}{\Delta x^2} \Big) - \frac{1}{\Delta x^2} 
 \Big( \frac{ u_i^{n+1} - 2 u_i^n + u_i^{n-1}}{\Delta t^2} \Big)\\ 
\end{multline*}
Here we can recognize finite differences of second order in time and space $\lbrack D_t D_t u \rbrack_i^n$ and $\lbrack D_x D_x u \rbrack_i^n$:
\begin{align*}
\frac{1}{6} \lbrack D_t D_t (D_x D_x u) \rbrack_i^n = c^2 \frac{1}{4} \lbrack D_x D_x u \rbrack_i^n - 
\frac{1}{\Delta x^2} \lbrack D_t D_t u \rbrack_i^n
\end{align*}
or equivalently,
\begin{align*}
\lbrack D_t D_t (u + \frac{1}{6} \Delta x^2 D_x D_x u) = c^2 \Delta x^2 /4 D_x D_x u \rbrack_i^n
\end{align*}

We have thus seen that the 1D wave equation $u_{tt} = c^2 u_{xx}$ can be solved with finite elements adjusted to have a finite difference behaviour. 
More simply, we manage to avoid solving a system of equations for each time step, and use an iterative method instead: 
this is preferable since system solving is time consuming and memory demanding. \newline 
The adjusted finite element method will still not be as good as the finite difference method because of a coefficient $\Delta x^2 /4$ which will have
a slowing effect on the method.
\end{description}

\end{document}          
