%%%%%%%%%%%%%%%%%%%%%%%%%%%%%%%%%%%%%%%%%
% Arsclassica Article
% LaTeX Template
% Version 1.1 (10/6/14)
%
% This template has been downloaded from:
% http://www.LaTeXTemplates.com
%
% Original author:
% Lorenzo Pantieri (http://www.lorenzopantieri.net) with extensive modifications by:
% Vel (vel@latextemplates.com)
%
% License:
% CC BY-NC-SA 3.0 (http://creativecommons.org/licenses/by-nc-sa/3.0/)
%
%%%%%%%%%%%%%%%%%%%%%%%%%%%%%%%%%%%%%%%%%

%----------------------------------------------------------------------------------------
%	PACKAGES AND OTHER DOCUMENT CONFIGURATIONS
%----------------------------------------------------------------------------------------

\documentclass[
10pt, % Main document font size
a4paper, % Paper type, use 'letterpaper' for US Letter paper
oneside, % One page layout (no page indentation)
%twoside, % Two page layout (page indentation for binding and different headers)
headinclude,footinclude, % Extra spacing for the header and footer
BCOR5mm, % Binding correction
]{scrartcl}

\input{structure.tex} % Include the structure.tex file which specified the document structure and layout

\hyphenation{Fortran hy-phen-ation} % Specify custom hyphenation points in words with dashes where you would like hyphenation to occur, or alternatively, don't put any dashes in a word to stop hyphenation altogether
\usepackage{float}
%----------------------------------------------------------------------------------------
%	TITLE AND AUTHOR(S)
%----------------------------------------------------------------------------------------

\title{Temperature front velocity in a two phase flow geothermal system} % The article title

\author{\spacedlowsmallcaps{Yapi Donatien Achou}} % The article author(s) - author affiliations need to be specified in the AUTHOR AFFILIATIONS block

\date{} % An optional date to appear under the author(s)

%----------------------------------------------------------------------------------------

\begin{document}

%----------------------------------------------------------------------------------------
%	HEADERS
%----------------------------------------------------------------------------------------

\renewcommand{\sectionmark}[1]{\markright{\spacedlowsmallcaps{#1}}} % The header for all pages (oneside) or for even pages (twoside)
%\renewcommand{\subsectionmark}[1]{\markright{\thesubsection~#1}} % Uncomment when using the twoside option - this modifies the header on odd pages
\lehead{\mbox{\llap{\small\thepage\kern1em\color{halfgray} \vline}\color{halfgray}\hspace{0.5em}\rightmark\hfil}} % The header style

\pagestyle{scrheadings} % Enable the headers specified in this block

%----------------------------------------------------------------------------------------
%	TABLE OF CONTENTS & LISTS OF FIGURES AND TABLES
%----------------------------------------------------------------------------------------

\maketitle % Print the title/author/date block

%\setcounter{tocdepth}{2} % Set the depth of the table of contents to show sections and subsections only

%\tableofcontents % Print the table of contents

%\listoffigures % Print the list of figures

%\listoftables % Print the list of tables



\section{Outline}
In this project the aim is to simulate the temperature front velocity in a two phase flow geothermal systems. This project is the natural continuation of \cite{Achou:2013}. In \cite{Achou:2013} the thermal front velocity of cold fluid injected into a single phase geothermal reservoir was computed. In this project the questions to be answer are: \\
1)How fast the thermal front velocity moves in a two phase flow geothermal reservoir \\
2)What are the affects of the cold front on phase fraction of steam and water in the production pipe\\
2)What are the effects of the cold front on the pressure drop along the production well\\
Electricity production in from geothermal energy rely on a sufficient amount of phase fraction from the production well. Injection of cold water increases the energy extracted from the system. However cooling of production wells can be induce during injection of colder fluid. 

\section{Software}
The software used in this project are OpenFoam, FEniCs, HOla. OpenFoam is an open source computational fluid dynamics software. Hola is a two phase flow well simulator developed in Iceland Goesurvey. 
In this project, OpenFoam will be used to simulate a two phase flow of geothermal fluid in a vertical and horizontal pipe. Hola will be used to simulate a steady state flow in the well and it output will served as the inlet boundary condition for the pipe. FEniCs will be used to simulate the thermal front velocity in the geothermal reservoir.

\section{Progression} 
The different step are:
1)literature review\\
2)simulation\\
3)writing\\

In the literature review the different models equations are selected and tested during the simulation.

\section{Timing}
An estimation of 350 hours will be devoted to this project. Starting time in set to January 2015.





%--------------------------------------------------------------oam --------------------------
%	BIBLIOGRAPHY
%----------------------------------------------------------------------------------------

\renewcommand{\refname}{\spacedlowsmallcaps{References}} % For modifying the bibliography heading

\bibliographystyle{unsrt}

\bibliography{sample.bib} % The file containing the bibliography

%----------------------------------------------------------------------------------------

\end{document}